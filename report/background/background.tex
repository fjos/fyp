\documentclass[../thesis.tex]{subfiles}
\begin{document}
\chapter{Background Theory}

\label{ch:background}

\section{Introduction}

Text of the Background.


    \subsection{OpenCL (Open Computing Language)} % (fold)
    \label{sub:opencl}
        The Open Computing Language (OpenCL) was originally developed and proposed by Apple Inc. as an open standard for parallel computing. After it's initial proposal by Apple, industry leaders worked together to create the technical specifications for the framework. The specifications were approved for and released publically by The Khronos Group on December 8, 2008\cite{opencl10pressrelease}.

        OpenCL is a framework for creating programs that execute across heterogeneous platforms (platforms consisting of more than one type of processor). It provides a standard application programming interface (API) for a wide variety of architectures, and today there are a large number of products that are OpenCL conformant\cite{khronosconformance}. As such, any code written in OpenCL can naturally target a large number of platforms, thus making it an attractive choice when developing parallel applications.  
    % subsection opencl (end)

    \subsection{OpenMPI} % (fold)
    \label{sub:openmpi}
        The Message Passing Interface (MPI) standard is a standard that ``addresses primarily the message-passing parallel programming model, in which data is moved from the address space of one process to that of another process through cooperative operations on each process''\cite{MPI-2.2}, or, in simpler terms, is a standard that primarily defines how multiple processes communicate with one another. 

        Additionally, it attempts to define an interface for a simple, portable, standard for communication. Some of the precepts laid down within the MPI standard is that communication should be reliable, that it should be usable on many platforms, and that it should be efficient.

        % Through the combined work of roughly 60 people from all across the world, a standard for message-passing was drafted and proposed in late 1992, and over time the standards have been corrected and refined. The most current standard is MPI-3.0.\cite{MPI-3.0}

        OpenMPI is an open source implementation of the MPI-2 standard, built from a merger between FT-MPI (University of Tennessee), LA-MPI (Los Alamos National Laboratory), and LAM/MPI (Indiana University) with the PACX-MPI team (University of Stuttgart) contributing. By combining these four implementations it was hoped that a fast, efficient MPI implementation could be built.\cite{openmpiwebsite} Additionally, it is supported by a wide variety of platforms, and is attractive in that regard as well.

        This implementation of MPI will provide this project's basis for data transfers and communication between networked GPUs.
    
    \subsection{CUDA and Close To Metal} % (fold)
    \label{sub:cuda_and_close_to_metal}
        CUDA (Compute Unified Device Architecture) is NVIDIA's proprietary solution to General-Purpose Computing on Graphics Processing Units (GPGPU). Although there could be a lengthy discussion about the differences between CUDA and OpenCL, research has shown that under a fair comparison there OpenCL and CUDA can achieve similar performance\cite{6047190}.

        As such, the chief difference is that CUDA enabled devices are available solely from NVIDIA\cite{cudagpus}, and the portability of OpenCL makes it much more attractive for distributed systems as it is not limited to a single brand of device.

        Close to Metal will only be mentioned briefly as it has been discontinued. It was, for a brief time, AMD Graphic Product Group's (then ATI's) solution to GPGPU. AMD has since switched to supporting OpenCL, thus increasing the number of platforms that utilize OpenCL as their primary method for GPGPU.
    % subsection cuda_and_close_to_metal (end)


    \subsection{Upcoming Developments} % (fold)
    \label{sub:upcoming_developments}
    As it is important to be aware of the state of the field, here we will discuss some of the upcoming developments and how they will affect the work.
        \subsubsection{OpenCL 2.0} % (fold)
        \label{ssub:opencl_2_0}
        OpenCL 2.0 is currently in development as the latest iteration of OpenCL. The finalized API Specification was released on November 14, 2013 and leads to some interesting changes. Shared virtual memory will be supported in OpenCL 2.0, which will allow for different kernel executions to share memory. Although this could probably be extended to a shared virtual memory layer between devices, without increased network speed this would probably be inefficient. As inefficiency is one of the things that will be fought against the most, it won't make much a difference to distributed implementations of OpenCL.
        % subsubsection opencl_2_0 (end)
    % subsection upcoming_developments (end)

    \subsection{Methodology Considered} % (fold)
    \label{sub:methodology_considered}
    In this section we'll briefly discuss some of the solutions considered, and their pros and cons.
        \subsubsection{Disguising Distance} % (fold)
        \label{ssub:disguising_distance}
            One possibility for implementing distributed OpenCL is to disguise the fact that compute devices are on separate nodes, and allow the OpenCL program to view all compute devices on a network as a local compute device. The advantage of this is that writing an OpenCL application for this type of an environment is trivial; all devices in a computer cluster will appear to be available locally, and all memory transfers are hidden from the user. This allows for a competent OpenCL programmer to write a program for a distributed environment with standard OpenCL API functions.

            There are, however, issues with this implementation. The chief of which is that it functions in a typical Master/Slave configuration; one machine controls what data and commands are passed, and to where. It does not easily allow for data transference between sister nodes, and thus the data flow paths are limited.

            It is a solution that is ideal for most tasks where the computation simply needs to be split between machines, and has been successfully implemented by researchers at Seoul National University\cite{Kim:2012:SOF:2304576.2304623}.
        % subsubsection disguising_distance (end)
        \subsubsection{Function Wrapping} % (fold)
        \label{ssub:function_wrapping}
            The method that will be used within this project is to create function wrappers for all the functions, defining origins and destinations for the data. Take, for example, you wish to load data from one machine onto another. Ordinarily this would involve multiple calls of MPI\_Send and MPI\_Recv in order to pass the size of the data, and then the data itself. The hope is to hide all of the sends and receives from the user, and allow them to use invocations very similar to OpenCL itself.

            This will require more thought on the programmers behalf, because they will need to understand how their data flows and where and when things need to execute, but it helps reduce how much thought they have to put into programming the data exchange.

            Both this and the previous solution are only applicable to local computing, where one can control all the machines that will be running the processes.
        % subsubsection function_wrapping (end)
        \subsubsection{High Throughput Computing} % (fold)
        \label{ssub:high_throughput_computing}
            A final thought will be given to High Throughput Computing, a type of computing meant more to take advantage of the computational power available to a large community. It's distributed computing that can more often be measured in months, less than high speed computational needs. It tackles the challenges of distributed computing by simply not having communication between individual devices, and simply returning results to the main system. Although some thought was given to implementing a framework for this, it is less measurable without a large community backing it, and proven solutions exist that have already implemented OpenCL\cite{boinc}. 
        % subsubsection high_throughput_computing (end)
    % subsection methodology_considered (end)

\end{document}
